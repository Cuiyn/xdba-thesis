
\chapter[本科生毕业设计论文撰写规范]{西安电子科技大学本科生毕业设计论文撰写规范}
\label{chap:requires}
\section{毕业设计(论文)的总体要求}
撰写论文应简明扼要,一般不少于15000字(外语专业可适当减少,但不得少于10000单词,且须全部用外语书写)。

\section{毕业设计(论文)的编写格式}
每一章、节的格式和版面要求整齐划一、层次清楚。其中:
\begin{itemize}
  \item 论文用纸:统一用A4纸,与论文封皮,任务书,工作计划,成绩考核表一致。
  \item 章的标题:如:``摘要''、``目录''、``第一章''、``附录''等,黑体,三号,居中排列。
  \item 节的标题:如:``2.1  认证方案''、``9.5  小结''等,宋体,四号,居中排列。
  \item 正文:中文为宋体,英文为``Times News Roman'',小四号。正文中的图名和表名,宋体,五号。
  \item 页眉:宋体五号,居中排列。左面页眉为论文题目,右面页眉为章次和章标题。页眉底划线的宽度为0.75磅。
  \item 页码:宋体小五号,排在页眉行的最外侧,不加任何修饰。
\end{itemize}

\section{毕业设计(论文)的前置部分}
毕业设计(论文)的前置部分包括封面、中英文摘要、目录等。
\subsection{封面及打印格式}
\begin{itemize}
  \item 学号:按照学校的统一编号,在右上角正确打印自己的学号,宋体,小四号,加粗。
  \item 题目:题目应和任务书的题目一致,黑体,三号。
  \item 学院、专业、班级、学生姓名和导师姓名职称等内容,宋体,小三号,居中排列。
\end{itemize}

\subsection{中英文摘要及关键词}
摘要是关于论文的内容不加注释和评论的简短陈述,具有独立性和自含性。它主要是
简要说明研究工作的目的、方法、结果和结论,重点说明本论文的成果和新见解。关键词
是为了文献标引工作从论文中选取出来用以表示全文主题内容信息的术语。
\begin{enumerate}
  \item 中文摘要,宋体小四号,一般为300字;英文摘要,``Times News Roman''字
体,小四号,一般为300个实词。摘要中不宜出现公式、非公用的符号、术语等。
  \item 每篇论文选取3 \~{} 5个关键词,中文为黑体小四号,英文为``Times News Roman''字体加粗,小四号。关键词排列在摘要的左下方一行,起始格式为:``\textbf{关键词}:
      ''和``\textbf{Keyword:}''。具体的各个关键词以均匀间隔排列,之间不加任何分隔符号。
\end{enumerate}

\section{目录}
按照论文的章、节、附录等前后顺序,编写序号、名称和页码。目录页排在中英文摘要之后,主体部分必
须另页右面开始,全文以右页为单页页码。

\section{毕业设计(论文)的主体部分}
毕业设计(论文)的主体部分包括引言(绪论)、正文、结论、结束语、致谢、参考文献。
\subsection{绪论}
作为论文的开端,简要说明作者所做工作的目的、范围、国内外进展情况、前人研究成果、
本人的设想、研究方法等。
\subsection{正文} 为毕业设计(论文)的核心部分,包括理论分析、数据资料、实验方法、结果、本人的论点和结
论等内容,还要附有各种有关的图表、照片、公式等。要求理论正确、逻辑清楚、层次分明、文字流畅、数据真实可
靠,公式推导和计算结果无误,图表绘制要少而精。
\begin{description}
  \item[图] 包括曲线图、示意图、流程图、框图等。图序号一律用阿拉伯数字分章依序编码,如:图1.3、图2.11。每一图应有简短确切的
      图名,连同图序号置于图的正下方。图中坐标上标注的符号和缩略词必须与正文中一致。
  \item[表] 包括分类项目和数据,一般要求分类项目由左至右横排,数据从上到下竖列。分类项目横排中必须标明符号或单位,竖列的数据栏中不宜出现``同上'' 、``同左''等类似词语,一律填写具体的数字或文字。表序号一律用阿拉伯数字分章依序编码,如:表2.5、表10.3。每一表应有简短确切的题名,
      连同表序号置于表的正上方。
  \item[公式] 正文中的公式、算式、方程式等必须编排序号,序号一律用阿拉伯数字分章依序编码,如:式(3-32)、式(6-21)。对于较长的公式,另行居中横排,只可在符号处(如:+、-、*、/、$<$、 $>$等)转行。公式序号标注于该式所在行(当有续行时,应标注于最后 一行)的最右边。连续性的公式在``=''处排列整齐。大于999的整数或多于三位的小数,一律用半个阿拉伯数字符的小间隔分开;小于1的数应将0置于小数点之前。
  \item[计量单位] 单位名称和符号的书写方式一律采用国际通用符号。
\end{description}

\subsection{结论}
是对主体的最终结论,应准确、完整、精炼。阐述作者创造性工作在本研究领域的地位和作用,对存在的问题和不足应给予客观的说明,也可提出进一步的设想。

\subsection{致谢}
对协助完成论文研究工作的单位和个人表示感谢。

\subsection{参考文献}
在学位论文中引用参考文献时,引出处右上角用方括号标注阿拉伯数字编排的序号(必须与参考文献一致)。参考文献的排列格
式分为:
\begin{description}
  \item[专著类的文献] [序号]  作者 . 专著名称.  版本. 出版地:出版者,出版年. 参考的页码。
  \item[期刊类的文献] 作者 . 文献名. 期刊名称.  年 , 月,  卷(期). 页码。
\end{description}
其中作者采用姓在前、名在后的形式。当作者超过三个时,只著录前三个人,其后
加``等''字即可。

\section{毕业设计(论文)的附录部分}
附录是作为学位论文主体的补充,包括下列内容:
\begin{enumerate}
  \item 正文中过于冗长的公式推导;
  \item 为读者阅读方便所需要的辅助性的数学工作或带有重复性的图表;
  \item 由于过分冗长而不宜在正文中出现的计算机程序清单;
  \item 对于一般读者并非必要阅读,但对本专业同行有参考价值的资料。
  \item 附录编于正文后,与正文连续编页码,每一附录均另页起。
  \item 附录依次用大写正体A,B,C……编序号,黑体,三号。如:附录A。
  \item 附录中的图、表、式、参考文献等与正文分开,用阿拉伯数字另行编序号,注意在数码前冠以附录的
      序码。如:图A1;表B2;式(C-3);文献[D5]。
\end{enumerate}
\section{毕业设计(论文)的打印规格}
论文正文页面和版面的设置规格:论文正文双面打印,为了便于装订、复制,要求每页纸的四周留有足够的空白边缘。以WORD97为例:

页面设置数据为:上3厘米、下2厘米、内侧3厘米、外侧2厘米;装订线 -- 1厘米;页眉  - 2厘米;  页脚 - 1厘米。

版面设置数据为:文字的行间距 - 1. 5倍 ;  公式的行间距 - 1. 5倍字符间距 - 标准;页码数据-对称页边距。

\section{毕业设计(论文)的装订说明}
毕业设计(论文)要求以A4纸的标准,按照下列顺序装订。外文资料翻译原文及译文另册装订,格式参照论文对应内容格式要求。

\begin{enumerate}
  \item 封面
  \item 任务书
  \item 工作计划
  \item 中期检查表
  \item 成绩考核登记表
  \item 中、外论文摘要
  \item 目录
  \item 引言
  \item 论文
  \item 结论
  \item 结束语
  \item 参考文献
  \item 附录
\end{enumerate}










