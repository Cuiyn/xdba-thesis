
\chapter{数学公式}
\label{chap:math}
准备好了,接下来我们就要领略到\TeX 强大之所在:数学符号和公式的排版。
本章所介绍的内容基本可以满足大部分人的需要。即便如此,也只是对此项功能的概括性的描述。
如果不能在此章中找到你所需要的排版学公式的方法,那么你可以在其它宏集中找到答案\cite{lshort-cn}。

\section{基本知识}
\LaTeX{}~使用一种特殊的模式来排版数学符号和公式(mathematics)。段落中的数学表达式应该
置于$\backslash$( 和$\backslash$), \$ 和\$ 或者$\backslash$begin{math} 和$\backslash$end{math} 之间。如,$c^2=a^2+b^2$~这就是一个很简单的例子。

对于较大的数学式子,最好的方法是使用显示式样来排版:将它们放于$\backslash$[~和$\backslash$]~或$\backslash$ begin{displaymath} 和$\backslash$ end{displaymath} 之间。这样排版出的公式是没有编号的。如果你希望\LaTeX{}~其添加编号的话,可以
使用equation 环境来达到这一目的。这是第一个例子,
\begin{displaymath}
e=mc^2
\end{displaymath}
下面是第二种带编号的例子,
\begin{equation}\label{eq:lim}
    \lim_{n \to \infty} \sum_{k=1}^n \frac{1}{k^2} = \frac{\pi^2}{6}
\end{equation}
从式(\ref{eq:lim})可以看出,在\LaTeX{}~中编辑数学公式是多么美好的一件事啊,引用也如此方便,根本
不用你操心,只管想问题就好了,最重要的是美观。

数学家们通常对使用什么样的符号非常挑剔:习惯上使用“空心粗体”(blackboard bold)来表示实数集合。这种字体可用amsfonts 或amssymb 宏包中的命命令$\backslash$mathbb
来得到。例如:
\begin{equation}\label{eq:sum}
x^2 \geq 0 \qquad \textrm{for all} x \in \mathbb{R}
\end{equation}

\section{常用数学公式}
在这一节中将介绍排版数学符号和公式的最重要的命令。数学模式中的命令仅对其后面第一个字符起作用。所以,
如果你希望某一命令作用于多个字符的话,那么你就必须将它们放置于括号中。平方根(square root)的输入演示:
\begin{equation}\label{eq:sq}
    \sqrt{c} = \sqrt{x^2+\sqrt[3]{y}}
\end{equation}

向量(Vectors)通常用上方有小箭头(arrow symbols)的变量表示。这可由vec 得到。另两个命令overrightarrow 和overleftarrow在定义从A 到B 的向量时非常有用。如$\vec{a}$和$\overrightarrow{AB}$是两个向量。

函数名通常用罗马字体正体排版,而不是像变量名一样用意大利体排版。因此,\LaTeX{}~提供命令来排版最重要的
一些函数名,如下式:
\begin{equation}\label{eq:sin}
    \lim_{x \to 0} \frac{\sin x}{x} = 1
\end{equation}

积分运算符(integral operator)用int 来生成。求和运算符(sum operator)由sum 生成。乘积运算符(product operator)由prod 生成。上限和下限用\^{} 和\_ 来生成,类似于上标和下标。参见式\eqref{eq:int}。
\begin{equation}
 \label{eq:int}
    f(x) = \int_0^{2\pi}{\sum x^2 + \prod_1^n x^3}dx
\end{equation}

多行公式:
\begin{align}
 \pi&= 3.14159265358979323\ldots \\   
  e &= 2.718281828\ldots
\end{align}

\section{定理和定义}
写数学文档时有可能需要一种方式来排版“引理”、“定义”、“公理”以及类似的结构。\LaTeX{}~为此提供了下述命令:
name 是短关键字,用于标识“定理”。text定义“定理”的真实名称,会在最终文件中打印出来。方括号中的选项是
任意的,可以用于指定“ 定理” 中使用的标号。counter 可以指定先前声明的“定理”的name。然后新“定理”会
按同样的顺序编号。section 指定“定理”编号所在的章节层次。
\begin{thm}
这是本包定义的第一个定理。可以用来测试。这是本包定义的第一个定理。可以用来测试。
这是本包定义的第一个定理。可以用来测试。这是本包定义的第一个定理。可以用来测试。
这是本包定义的第一个定理。可以用来测试。
\end{thm}

\begin{thm}
这是本包定义的第二个定理。可以用来测试。这是本包定义的第二个定理。可以用来测试。
\begin{equation}
    \lim_{n \to \infty} \sum_{k=1}^n \frac{1}{k^2} = \frac{\pi^2}{6}
\end{equation}
这是本包定义的第二个定理。可以用来测试。这是本包定义的第二个定理。可以用来测试。
\end{thm}

\begin{algo}
这是本包定义的第一个算法。可以用来测试。这是本包定义的第一个算法。可以用来测试。
这是本包定义的第一个算法。可以用来测试。
这是本包定义的第一个算法。可以用来测试。
这是本包定义的第一个算法。可以用来测试。
这是本包定义的第一个算法。可以用来测试。

\end{algo}





