
\chapter{引言}
\label{chap:introduction}

本文介绍了西电版的\LaTeX{}~本科毕业设计论文模板,该模板是基于\CTeX{}~中
文宏包开发,指在为西安电子科技大学的本科毕业生提供一个简单、专业、有效的
排版工具,且该版本不打算加入研究生毕业论文和博士生毕业论文,因为定制模板
也是一个很复杂的事情,如果有可能的话,后期可能继续单独写研究生和博士生的
\LaTeX{}~模板。作者本着为西电同学服务的原则开发,并不承担一切有关责任与
义务,如维护、更新等,但欢迎提交~\texttt{BUG}~。祝西安电子科技大学的同学前程似锦。

本模板严格按照西安电子科技大学教务处教学实践中心下发的最新2010本科生毕业设计
论文要求设计制作,经过初步测试,已经完全符合要求,但是这不能保证完全无误,所
以望同学们使用时仔细慎重,如一经发现任何不符合要求之处,请联系\href{mailto:xuejilong@gmail.com}{xuejilong@gmail.com}
~进行修改,本项目的主页是\href{code.google.com/p/xdba-thesis}{code.google.com/p/xdba-thesis}~,欢迎下载最新
 版本。

\LaTeX{}~是一种高质量的排版工具,编写宏包目的是简化学位论文的撰写,使得论文
作者可以将精力集中到论文的内容上而不是浪费在版面设置上。同时宏包在符合学位
论文撰写要求的基础上尽可能地进行美化,其中还参考了出版界的一些排版规范。所以
极力推介大家使用。

\section{系统环境}
由于本模板未在其它任何环境下试用,所以提供开发环境,以便正常使用。该宏包是基于
最新的\CTeX v2.9.0.152~ 中文套装\cite{site:ctex}环境下开发编写,底层支持~CCT~和\allowbreak~CJK~两种中文~\LaTeX{}~
系统。所以可以在该环境在正常使用和二次修改。

按常规,该包宏包可以在目前大多数的~\TeX{}~系统中使用,例如~C\TeX{}、~MiK\TeX{}、
~te\TeX{}、~fp\TeX{}。

此外,~\texttt{XDBAthesis}~宏包还使用了宏包~amsmath、~amsthm、~amsfonts、
~amssymb、~bm~、~hyperref~和~caption2~。目前大多数的~\TeX{}~系统中都包含有这些宏包。
中包含了以上列出的各种宏包,用户无需额外的设置即可使用。

\section{完善与更新}

XDBAthesis宏包的最新版本可以从~\url{http://code.google.com/p/xdba-thesis}~网站下载。
该宏包包含两个文件:~\texttt{XDBAthesis.cls}~和~\texttt{XDBAthesis.cfg}。
简单方便的安装方法是将宏包文件和学位论文~\texttt{.tex}~文件放置在同一目录下。

同时,宏包还提供了一个使用模板,也就是这份说明文档的源文件。用户可以通过修改
这个模板来编写自己的学位论文。

由于该包首次制作,难免有问题,所以有使用发现的,其及时与作者联系,或者到上述网站参与
维护更新。


\section{宏包定制}

\texttt{XDBAthesis}~宏包的设置都保存在~\texttt{XDBAthesis.cfg}~文件中。
用户可以在\allowbreak~\texttt{.tex}~中通过宏包提供的命令修改设置。对于常用的设置修改,
如学校、专业名称等,可以直接在~\texttt{XDBAthesis.cfg}~文件中进行。但极不推介
修改,因为该配置是作者精心修改编辑,完全符合我校目前毕业设计论文要求,只需要
在模板页中写自己的相关信息即可,为你的毕业论文省去一大笔时间。

\section{宏包版权}
大多数软件许可证决意剥夺你的共享和修改软件的自由。对比之下,GNU通用公共许可证
力图保证你的共享和修GPL改自由软件的自由。保证自由软件对所有用户是自由的。GPL适
用于大多数自由软件基金会的软件,以及由使用这些软件而承担义务的作者所开发的软件。
(自由软件基金会的其他一些软件受GNU库通用许可证的保护)。你也可以将它用到你的程序中。
当我们谈到自由软件(free software)时,我们指的是自由而不是价格。
  
我们的GNU通用公共许可证决意保证你有发布自由软件的自由(如果你愿意,你可以对此项服
务收取一定的费用);保证你能收到源程序或者在你需要时能得到它;保证你能修改软件或将
它的一部分用于新的自由软件;而且还保证你知道你能做这些事情。为了保护你的权利,
我们需要作出规定:禁止任何人不承认你的权利,或者要求你放弃这些权利。如果你修改
了自由软件或者发布了软件的副本,这些规定就转化为你的责任。例如,如果你发布这样一
个程序的副本,不管是收费的还是免费的,你必须将你具有的一切权利给予你的接受者;
你必须保证他们能收到或得到源程序;并且将这些条款给他们看,使他们知道他们有这样的权利。

\section{问题反馈}

用户在使用中遇到问题或者需要增加某种功能,都可以和作者联系:

\begin{center}
薛继龙(jlxue) \quad \href{mailto:xuejilong@gmail.com}{xuejilong@gmail.com}\quad
\href{www.jlxue.cn}{www.jlxue.cn}
\end{center}

欢迎大家反馈自己的使用情况,能为西电本科生的作出一点点的贡献,也祝西安电子科技
大学的同学前程似锦。
